\section*{\label{sec:transposition}Parallel matrix transposition}

The following section describes the implementation of a parallel matrix transposition algorithm with and %
without blocks using OpenMPI. The algorithm is implemented in \texttt{C}, and the source code is available at %
in the \texttt{src} directory of the project repository.
Instead of developing distinct functions for serial and parallel code, the decision was made to implement %
unique algorithms adaptable to the number of available processors. This approach not only maintains code %
cleanliness but also streamlines the testing process. Furthermore, it facilitates the execution of the serial %
code while accounting for instructions introducing communication overhead, thereby enabling a more precise %
comparison between the versions.

\setlength{\abovecaptionskip}{-5pt} % Adjust the value as needed
\begin{wrapfigure}{r}{0.4\textwidth}
    \caption{\label{image:pointers-to-pointers}Array of pointers to arrays}
    \centering
    \includegraphics[width=0.5\linewidth]{pointers-to-pointers.tex}
\end{wrapfigure}

Two variants of the transposition algorithm were considered, delineating differences in how the matrix is %
partitioned among processors. The initial approach involves the segmentation of the matrix into square blocks, %
while the second approach employs rectangular blocks. In both instances, the matrix is treated as a one-dimensional %
array due to empirical findings suggesting that MPI handles this structure more efficiently than two-dimensional matrices. %
(Figure \ref{image:pointers-to-pointers}). %

Delving into specifics, the first version subdivides the matrix into blocks of size $n * n$, where $n$ is equal %
to the matrix size divided by the square root of the number of processors. This ensures equitable distribution, %
where each processor transposes an equal number of rows and columns. However, this introduces the prerequisite %
that the number of processors must be a perfect square (Figure \ref{image:matrix-as-array}).

\begin{figure}[h]
    \caption{\label{image:matrix-as-array}Matrix as a 1D array}
    \centering
    \includegraphics[width=0.7\linewidth]{matrix-as-array.tex}
\end{figure}

Conversely, the second version divides the matrix into blocks of size $f * c$, where $f$ corresponds to the matrix %
size divided by the number of processors, and $c$ is equal to the size of the matrix itself. The only limitation %
of this approach is that the number of processors must be a divisor of the matrix size (Figure \ref{image:transpose-blocks}).

\begin{figure}[h]
    \caption{\label{image:transpose-blocks}Divide into $n$ blocks, each transposed individually}
    \centering
    \includegraphics[width=0.7\linewidth]{transpose-blocks.tex}
\end{figure}

In both scenarios, each processor receives a matrix block, performs the transposition, and subsequently transmits the %
result to the master processor. The master processor orchestrates the reconstruction of the final matrix, ensuring the %
appropriate alignment of blocks.

In the version with blocks of the algorithm, the original matrix, still represented as a one-dimensional array, is divided %
into blocks of equal size, each corresponding to the number of processors. Each of these blocks is further subdivided %
into the same number of sub-blocks and distributed among the processors, which subsequently transposed them.
This is achieved by using the \texttt{MPI\_Scatterv} function, which divides the large block into equal-sized blocks and distributes them.
The transposed results are then transmitted back to the master processor using the \texttt{MPI\_Gatherv} function (the inverse of %
\texttt{MPI\_Scatterv}), where they are assembled into the large transposed block.
Before moving on to the next block, the transposed block is appropriately positioned to construct the final transposed %
matrix, also stored as a one-dimensional array. 
Also in this scenario, the algorithm introduces specific constraints. The major requirement is that the number of processors must be %
a perfect square to facilitate an equitable distribution of elements. Additionally, the matrix size must be at least twice the number %
of processors to ensure that each processor receives at least one element from each row and column of the matrix, thereby guaranteeing %
an even distribution of computational workload. This safeguards against scenarios where the matrix size might be insufficient to fully %
utilize the available processing units, potentially resulting in the underutilization of computational resources. Thus, these constraints %
collectively aim to optimize the parallel execution of the algorithm by promoting balanced data distribution among processors and %
maximizing the utilization of available computing resources.

\section*{Result analysis}

The performance of the parallel algorithm was evaluated considering both strong and weak scaling. %
The parameters used for the evaluation are the speedup, the efficiency gains, and the bandwidth. The formulas %
used to calculate these parameters are as follows:

\begin{equation}
    \label{eq:speedup}
    \text{Speedup} = \frac{\text{Serial Run Time}}{\text{Parallel Run Time}} \quad \text{[times]}
\end{equation}

\begin{equation}
    \label{eq:efficiency}
    \text{Efficiency} = \frac{\text{Speedup}}{\text{Number of Processors}} * 100 \quad \text{[\%]}
\end{equation}

\begin{equation}
    \label{eq:bandwidth}
    \text{Bandwidth} = \frac{B_r * B_w * \text{Times} * \text{\textit{Float} size}}{\text{Run Time} * 10^9} = 
    \begin{cases}
        \frac{\text{Matrix Size}^2 * 3 * \text{\textit{Float} Size}}{\text{Run Time} * 10^9}, & \text{without blocks}\\
        \frac{\text{Matrix Size}^2 * 4 * \text{\textit{Float} Size}}{\text{Run Time} * 10^9}, & \text{with blocks}
    \end{cases} \quad
    \text{[GB/s]}
\end{equation}

where $B_r$ and $B_w$ are the number of bytes read and written, respectively, and \textit{Float} size is the size of a %
\texttt{float} variable in bytes. \texttt{Times} is the number of times the matrix is read and written during the execution %
of the algorithm. In the case of the algorithm without blocks, the matrix is read and written three times: once to subdivide %
the matrix, once to transpose it, and once to recompose the final matrix. In the case of the algorithm with blocks, the matrix %
is read and written once more than the algorithm without blocks, caused by the additional step of subdividing into smaller blocks.

I encountered challenges in obtaining results for the missing values in the following tables. The execution faced delays, eventually %
leading to termination by the system when the wall time was reached. This issue likely arises due to insufficient available memory %
on the assigned nodes.

The results obtained are summarized in Tables \ref{table:strong-scaling} and \ref{table:weak-scaling}. %

\begin{tabularx}{\textwidth}{@{} c c c c c c @{}}
    \caption{\label{table:strong-scaling}Strong scaling}\\
    \toprule
        \textbf{Size} & \textbf{Cores} & \multicolumn{2}{c}{\textbf{Normal}} & \multicolumn{2}{c}{\textbf{Blocks}}\\
        \cmidrule(lr){3-4}\cmidrule(lr){5-6}
        & & \textbf{Run Time [s]} & \textbf{Bandwidth [GB/s]} & \textbf{Run Time [s]} & \textbf{Bandwidth [GB/s]} \\
    \midrule
    \endhead
        \multirow{4}{*}{512} & 1	& 0.002226	& 1,413	& 0.005520	& 0.760 \\
                             & 4	& 0.002251	& 1,397	& 0.003363	& 1,247 \\
                             & 16	& 0.005359	& 0,587	& 0.003376	& 1,242 \\
                             & 64	& 0.014420	& 0,218	& 0.009164	& 0,458 \\
    \midrule
        \multirow{4}{*}{2048} & 1   & 0.064612 & 0.779 & 0.079998 & 0.839 \\
                              & 4   & 0.026691 & 1.886 & 0.044276 & 1.516 \\
                              & 16  & 0.025379 & 1.983 & 0.041785 & 1.606 \\
                              & 64  & 0.083068 & 0.606 & 0.049830 & 1.347 \\
        
    \midrule
        \multirow{4}{*}{8192} & 1   & 3.729533 & 0.216 & 2.515364 & 0.427 \\
                              & 4   & 1.019834 & 0.790 & 0.901501 & 1.191 \\
                              & 16  & 1.176419 & 0.685 & 0.627510 & 1.711 \\
                              & 64  & 1.462280 & 0.551 & 0.765071 & 1.403 \\
        
    \midrule
        \multirow{4}{*}{32768} & 1  & 57.112991 & 0.226 & 67.101906 & 0.256 \\
                               & 4  & -         & -     & 24.95976 & 0.688 \\
                               & 16 & -         & -     & 11.06617 & 1.552 \\
                               & 64 & -         & -     & 11.33467 & 1.516 \\
        
    \bottomrule
\end{tabularx}
\begin{tabularx}{\textwidth}{@{} c c c c c c @{}}
    \caption{\label{table:weak-scaling}Weak scaling}\\
    \toprule
        \textbf{Cores} & \textbf{Size} & \multicolumn{2}{c}{\textbf{Normal}} & \multicolumn{2}{c}{\textbf{Blocks}}\\
        \cmidrule(lr){3-4}\cmidrule(lr){5-6}
        & & \textbf{Run Time [s]} & \textbf{Bandwidth [GB/s]} & \textbf{Run Time [s]} & \textbf{Bandwidth [GB/s]} \\
    \midrule
    \endhead
        \multirow{4}{*}{1} & 64  &  &  &  &  \\
                           & 128 &  &  &  &  \\
                           & 256 &  &  &  &  \\
                           & 512 &  &  &  &  \\
    \midrule
        \multirow{4}{*}{4} & 256  &  &  &  &  \\
                           & 512  &  &  &  &  \\
                           & 1024 &  &  &  &  \\
                           & 2048 &  &  &  &  \\
    \midrule
        \multirow{4}{*}{16} & 1024 &  &  &  &  \\
                            & 2048 &  &  &  &  \\
                            & 4096 &  &  &  &  \\
                            & 8192 &  &  &  &  \\
    \midrule
        \multirow{4}{*}{64} & 4096  &  &  &  &  \\
                            & 8192  &  &  &  &  \\
                            & 16384 &  &  &  &  \\
                            & 32768 &  &  &  &  \\
    \bottomrule
\end{tabularx}

In the weak scaling plot (Plot \ref{plot:weak-scale}), can be observed that despite the constant workload per processor, the execution time tends to increase as the number of processors grows. %
This phenomenon is attributed to the fact that an increase in processors also augments the number of communications required for processor synchronization, %
becoming more frequent and costly.\\
In the strong scaling one (Plot \ref{plot:strong-scale}), on the other hand, as the number of processors increases, the execution time tends to decrease, although not linearly. %
This behaviour is attributed to the same reason as in weak scaling. These results are primarily supported by efficiency graphs, demonstrating a decreasing %
trend in efficiency as the number of processors increases.\\
Plot \ref{plot:efficiency} shows that the efficiency of the block-based program steadily decreases with an increasing number of processors, while the algorithm without blocks %
exhibits a more irregular trend, experiencing a sharp efficiency decline already at 4 processors.\\
Contrary to expectations, the bandwidth graph (Plot \ref{plot:bandwidth}) shows irregular behaviour for the blockless algorithm with a noticeable decline at 64 processors. In contrast, %
the block-based algorithm exhibits a more regular trend in line with expectations. Nevertheless, a decline is observed with a matrix size of 512 due to its %
reduced dimensions, preventing optimal utilization of available resources.\\
The speedup graphs (Plot \ref{plot:speedup}) summarize the aforementioned observations, revealing that the blockless algorithm, despite an initial peak in performance at 4 processors, %
tends to decrease with an increasing number of CPUs. Meanwhile, the block-based algorithm displays a more consistent trend, experiencing a peak in %
performance at 16 processors for the larger matrix and maintaining a steady trend for others.\\
From the strong scaling graph, can be further observed that beyond 4 processors for the blockless version and 16 processors for the block-based version, the %
execution time tends to stabilize. This phenomenon could be attributed to the fact that as the number of processors increases, the execution time may %
decrease, but delays introduced by communication between processors offset the saved time.

From all these data, can be asserted that MPI may not be the optimal approach for the direct parallelization of algorithms. %
Indeed, MPI serves as a framework for inter-process communication rather than algorithm parallelization. MPI is better suited %
for applications that demand process communication, such as matrix multiplication, where computations are performed in parallel, %
but inter-process communication is essential for the summation of partial results. In such cases, inter-process communication %
is substantial and time-consuming, making MPI less suitable as the preferred framework.

\begin{figure}[h!tb]%
    \centering
    \captionsetup{type=plot}
    \caption{\label{plot:strong-scale}Strong scale graphs by size of matrices}
    \begin{tikzpicture}%
        \begin{groupplot}[
                group style={
                    group size=2 by 1,
                    group name=plot,
                    xlabels at=edge bottom,
                    ylabels at=edge left,
                    vertical sep=1.5cm,
                    horizontal sep=1.5cm,
                }, 
                width={0.45\linewidth},
                xlabel={CPUs},
                ylabel={Time [s]},
                xtick={1, 4, 16, 64},
                legend to name=legend,
                legend columns=-1,
                cycle multiindex* list={
                    color list
                        \nextlist
                    marks
                        \nextlist
                },
            ]
            \nextgroupplot[
                title={Normal},
                xmode=log, 
                log ticks with fixed point,
            ]

                \addplot
                    coordinates {
                        (1, 0.002226)
                        (4, 0.002251)
                        (16, 0.005359)
                        (64, 0.014420)
                    };
        
                \addplot
                    coordinates {
                        (1, 0.064612)
                        (4, 0.026691)
                        (16, 0.025379)
                        (64, 0.083068)
                    };
                
                \addplot
                    coordinates {
                        (1, 3.729533)
                        (4, 1.019834)
                        (16, 1.176419)
                        (64, 1.462280)
                    };
                    
            \nextgroupplot[
                title={Blocks},
                xmode=log, 
                log ticks with fixed point,
            ]
                \addplot
                    coordinates {
                        (1, 0.005520)
                        (4, 0.003363)
                        (16, 0.003376)
                        (64, 0.009164)
                    };
                
                \addplot
                    coordinates {
                        (1, 0.079998)
                        (4, 0.044276)
                        (16, 0.041785)
                        (64, 0.049830)
                    };

                \addplot
                    coordinates {
                        (1, 4.186348)
                        (4, 0.901501)
                        (16, 0.627510)
                        (64, 0.765071)
                    };
                
                \addplot
                    coordinates {
                        (1, 67.101906)
                        (4, 24.95976)
                        (16, 11.06617)
                        (64, 11.33467)
                    }; 

                \legend{512, 2048, 8192, 32768}
        \end{groupplot}
    \end{tikzpicture}
    \ref{legend}
\end{figure}
\begin{figure}[h!tb]%
    \centering
    \captionsetup{type=plot}
    \caption{\label{plot:weak-scale}Weak scale graphs by matrix size-CPUs ratio}
    \begin{tikzpicture}%
        \begin{groupplot}[
                group style={
                    group size=2 by 1,
                    group name=plot,
                    xlabels at=edge bottom,
                    ylabels at=edge left,
                    vertical sep=1.5cm,
                    horizontal sep=1.5cm,
                }, 
                width={0.45\linewidth},
                xlabel={CPUs},
                ylabel={Time [s]},
                xtick={1,4,16,64},
                legend to name=legend,
                legend columns=-1,
                cycle multiindex* list={
                    color list
                        \nextlist
                    marks
                        \nextlist
                },
            ]
            \nextgroupplot[
                title={Normal},
                xmode=log, 
                log ticks with fixed point,
            ]

            \addplot
                coordinates {
                    (1, 0.000034)
                    (4, 0.001141)
                    (16, 0.013206)
                    (64, 0.218325)
                };

            \addplot
                coordinates {
                    (1, 0.000111)
                    (4, 0.002095)
                    (16, 0.039099)
                    (64, 0.650468)
                };
        
            \addplot
                coordinates {
                    (1, 0.000441)
                    (4, 0.006750)
                    (16, 0.107758)
                    (64, 5.574380)
                };

            \addplot
                coordinates {
                    (1, 0.002263)
                    (4, 0.025396)
                    (16, 0.659142)
                };

            \nextgroupplot[
                title={Blocks},
                xmode=log, 
                log ticks with fixed point,
            ]
            \addplot
                coordinates {
                    (1, 0.000097)
                    (4, 0.001797)
                    (16, 0.010417)
                    (64, 0.171367)
                };

            \addplot
                coordinates {
                    (1, 0.000336)
                    (4, 0.003130)
                    (16, 0.042607)
                    (64, 0.684192)
                };
        
            \addplot
                coordinates {
                    (1, 0.001309)
                    (4, 0.010760)
                    (16, 0.158777)
                    (64, 2.789386)
                };

            \addplot
                coordinates {
                    (1, 0.005754)
                    (4, 0.043814)
                    (16, 0.636953)
                    (64, 11.343099)
                };

                \legend{64/1, 128/1, 256/1, 512/1}
        \end{groupplot}
    \end{tikzpicture}
    \ref{legend}
\end{figure}

\begin{figure}[h!tb]%
    \centering
    \captionsetup{type=plot}
    \caption{\label{plot:efficiency}Efficiency graphs by size of matrices}
    \begin{tikzpicture}%
        \begin{groupplot}[
                group style={
                    group size=2 by 1,
                    group name=plot,
                    xlabels at=edge bottom,
                    ylabels at=edge left,
                    vertical sep=1.5cm,
                    horizontal sep=1.5cm,
                }, 
                width={0.45\linewidth},
                xlabel={CPUs},
                ylabel={Efficiency [\%]},
                xtick={1, 4, 16, 64},
                legend to name=legend,
                legend columns=-1,
                cycle multiindex* list={
                    color list
                        \nextlist
                    marks
                        \nextlist
                },
            ]
            \nextgroupplot[
                title={Normal},
                xmode=log,
                log ticks with fixed point,
            ]

                \addplot
                    coordinates {
                        (1, 100)
                        (4, 24.72)
                        (16, 2.60)
                        (64, 0.24)
                    };
        
                \addplot
                    coordinates {
                        (1, 100)
                        (4, 60.52)
                        (16, 15.91)
                        (64, 1.22)
                    };

                \addplot
                    coordinates {
                        (1, 100)
                        (4, 91.43)
                        (16, 19.81)
                        (64, 3.99)
                    };

            \nextgroupplot[
                title={Blocks},
                xmode=log, 
                log ticks with fixed point,
            ]
                \addplot
                    coordinates {
                        (1, 100)
                        (4, 41.03)
                        (16, 10.22)
                        (64, 0.94)
                    };
                
                \addplot
                    coordinates {
                        (1, 100)
                        (4, 45.17)
                        (16, 11.97)
                        (64, 2.51)
                    };

                \addplot
                    coordinates {
                        (1, 100)
                        (4, 69.75)
                        (16, 25.05)
                        (64, 5.14)
                    };
                
                \addplot
                    coordinates {
                        (1, 100)
                        (4, 67.21)
                        (16, 37.90)
                        (64, 9.25)
                    }; 

                \legend{512, 2048, 8192, 32768}
        \end{groupplot}
    \end{tikzpicture}
    \ref{legend}
\end{figure}
\begin{figure}[h!tb]%
    \centering
    \captionsetup{type=plot}
    \caption{\label{plot:speedup}Speedup graphs by size of matrices}
    \begin{tikzpicture}%
        \begin{groupplot}[
                group style={
                    group size=2 by 1,
                    group name=plot,
                    xlabels at=edge bottom,
                    ylabels at=edge left,
                    vertical sep=1.5cm,
                    horizontal sep=1.5cm,
                }, 
                width={0.45\linewidth},
                xlabel={CPUs},
                ylabel={Speedup [$\times$]},
                xtick={1, 4, 16, 64},
                legend to name=legend,
                legend columns=-1,
                cycle multiindex* list={
                    color list
                        \nextlist
                    marks
                        \nextlist
                },
            ]
            \nextgroupplot[
                title={Normal},
                xmode=log, 
                log ticks with fixed point,
            ]

                \addplot
                    coordinates {
                        (1, 1.00)
                        (4, 0.99)
                        (16, 0.42)
                        (64, 0.15)
                    };
        
                \addplot
                    coordinates {
                        (1, 1.00)
                        (4, 2.42)
                        (16, 2.55)
                        (64, 0.78)
                    };

                \addplot
                    coordinates {
                        (1, 1.00)
                        (4, 3.66)
                        (16, 3.17)
                        (64, 2.55)
                    };

            \nextgroupplot[
                title={Blocks},
                xmode=log, 
                log ticks with fixed point,
            ]
                \addplot
                    coordinates {
                        (1, 1)
                        (4, 1.64)
                        (16, 1.64)
                        (64, 0.60)
                    };
                
                \addplot
                    coordinates {
                        (1, 1.00)
                        (4, 1.81)
                        (16, 1.91)
                        (64, 1.61)
                    };

                \addplot
                    coordinates {
                        (1, 1.00)
                        (4, 2.79)
                        (16, 4.01)
                        (64, 3.29)
                    };
                
                \addplot
                    coordinates {
                        (1, 1.00)
                        (4, 2.69)
                        (16, 6.06)
                        (64, 5.92)
                    }; 

                \legend{512, 2048, 8192, 32768}
        \end{groupplot}
    \end{tikzpicture}
    \ref{legend}
\end{figure}
\begin{figure}[h!tb]%
    \centering
    \captionsetup{type=plot}
    \caption{\label{plot:bandwidth}Performance graphs by size of matrices}
    \begin{tikzpicture}%
        \begin{groupplot}[
                group style={
                    group size=2 by 1,
                    group name=plot,
                    xlabels at=edge bottom,
                    ylabels at=edge left,
                    vertical sep=1.5cm,
                    horizontal sep=1.5cm,
                }, 
                width={0.45\linewidth},
                xlabel={CPUs},
                ylabel={Bandwidth [GB/s]},
                xtick={1, 4, 16, 64},
                legend to name=legend,
                legend columns=-1,
                cycle multiindex* list={
                    color list
                        \nextlist
                    marks
                        \nextlist
                },
            ]
            \nextgroupplot[
                title={Normal},
                xmode=log, 
                log ticks with fixed point,
            ]

                \addplot
                    coordinates {
                        (1, 1.413)
                        (4, 1.397)
                        (16, 0.587)
                        (64, 0.218)
                    };
        
                \addplot
                    coordinates {
                        (1, 0.779)
                        (4, 1.886)
                        (16, 1.983)
                        (64, 0.606)
                    };

                \addplot
                    coordinates {
                        (1, 0.216)
                        (4, 0.790)
                        (16, 0.685)
                        (64, 0.551)
                    };

            \nextgroupplot[
                title={Blocks},
                xmode=log, 
                log ticks with fixed point,
            ]
                \addplot
                    coordinates {
                        (1, 0.760)
                        (4, 1.247)
                        (16, 1.242)
                        (64, 0.458)
                    };
                
                \addplot
                    coordinates {
                        (1, 0.839)
                        (4, 1.516)
                        (16, 1.606)
                        (64, 1.347)
                    };

                \addplot
                    coordinates {
                        (1, 0.427)
                        (4, 1.191)
                        (16, 1.711)
                        (64, 1.403)
                    };
                
                \addplot
                    coordinates {
                        (1, 0.256)
                        (4, 0.688)
                        (16, 1.552)
                        (64, 1.516)
                    }; 

                    \legend{512, 2048, 8192, 32768}
        \end{groupplot}
    \end{tikzpicture}
    \ref{legend}
\end{figure}
