\section*{\label{sec:transposition}Parallel matrix transposition}

The following section describes the implementation of a parallel matrix transposition algorithm with and %
without blocks using OpenMPI. The algorithm was implemented in \texttt{C}, and the source code is available at %
in the \texttt{src} directory of the project repository.
Instead of developing distinct functions for serial and parallel code, the decision was made to implement %
a unique algorithms adaptable to the number of available processors. This approach not only maintains code %
cleanliness but also streamlines the testing process. Furthermore, it facilitates the execution of the serial %
code while accounting for instructions introducing communication overhead, thereby enabling a more precise %
comparison between the versions.

\setlength{\abovecaptionskip}{-5pt} % Adjust the value as needed
\begin{wrapfigure}{r}{0.4\textwidth}
    \caption{\label{image:pointers-to-pointers}Array of pointers to arrays}
    \centering
    \includegraphics[width=0.5\linewidth]{pointers-to-pointers.tex}
\end{wrapfigure}

Two variants of the transposition algorithm were considered, delineating differences in how the matrix is %
partitioned among processors. The initial approach involves the segmentation of the matrix into square blocks, %
while the second approach employs rectangular blocks. In both instances, the matrix is treated as a one-dimensional %
array due to empirical findings suggesting that MPI handles this structure more efficiently than two-dimensional matrices. %
(Figure \ref{image:pointers-to-pointers}). %

Delving into specifics, the first version subdivides the matrix into blocks of size $n * n$, where $n$ is equal %
to the matrix size divided by the square root of the number of processors. This ensures equitable distribution, %
where each processor transposes an equal number of rows and columns. However, this introduces the prerequisite %
that the number of processors must be a perfect square (Figure \ref{image:matrix-as-array}).

\begin{figure}[h]
    \caption{\label{image:matrix-as-array}Matrix as a 1D array}
    \centering
    \includegraphics[width=0.7\linewidth]{matrix-as-array.tex}
\end{figure}

Conversely, the second version divides the matrix into blocks of size $f * c$, where $f$ corresponds to the matrix %
size divided by the number of processors, and $c$ is equal to the size of the matrix itself. The only limitation %
of this approach is that the number of processors must be a divisor of the matrix size (Figure \ref{image:transpose-blocks}).

\begin{figure}[h]
    \caption{\label{image:transpose-blocks}Divide in $n$ blocks, each transposed individually}
    \centering
    \includegraphics[width=0.7\linewidth]{transpose-blocks.tex}
\end{figure}

In both scenarios, each processor receives a matrix block, performs the transposition, and subsequently transmits the %
result to the master processor. The master processor orchestrates the reconstruction of the final matrix, ensuring the %
appropriate alignment of blocks.

In the version with blocks of the algorithm, the original matrix, still represented as a one-dimensional array, is divided %
into blocks of equal size, each corresponding to the number of processors. Each of these blocks is further subdivided %
into the same number of sub-blocks and distributed among the processors, which subsequently transposed them.
This is achieved by using the \texttt{MPI\_Scatterv} function, which divides the large block into equal-sized blocks and distributes them.
The transposed results are then transmitted back to the master processor using the \texttt{MPI\_Gatherv} function (the inverse of %
\texttt{MPI\_Scatterv}), where they are assembled into the large transposed block.
Before moving on to the next block, the transposed block is appropriately positioned to construct the final transposed %
matrix, also stored as a one-dimensional array. 
Also in this scenario, the algorithm introduces specific constraints. The major requirement is that the number of processors must be %
a perfect square to facilitate an equitable distribution of elements. Additionally, the matrix size must be at least twice the number %
of processors to ensure that each processor receives at least one element from each row and column of the matrix, thereby guaranteeing %
an even distribution of computational workload. This safeguards against scenarios where the matrix size might be insufficient to fully %
utilize the available processing units, potentially resulting in underutilization of computational resources. Thus, these constraints %
collectively aim to optimize the parallel execution of the algorithm by promoting balanced data distribution among processors and %
maximizing the utilization of available computing resources.

\section*{Result analysis}

The performance of the parallel algorithm was evaluated taking into account both strong and weak scaling. %
The parameters used for the evaluation are the speedup, the efficiency gains and the bandwidth. The formulas %
used to calculate these parameters are as follows:

\begin{equation}
    \label{eq:speedup}
    \text{Speedup} = \frac{\text{Serial Run Time}}{\text{Parallel Run Time}} \quad \text{[times]}
\end{equation}

\begin{equation}
    \label{eq:efficiency}
    \text{Efficiency} = \frac{\text{Speedup}}{\text{Number of Processors}} * 100 \quad \text{[\%]}
\end{equation}

\begin{equation}
    \label{eq:bandwidth}
    \text{Bandwidth} = \frac{B_r * B_w * \text{Times} * \text{\textit{Float} size}}{\text{Run Time} * 10^9} = 
    \begin{cases}
        \frac{\text{Matrix Size}^2 * 3 * \text{\textit{Float} Size}}{\text{Run Time}}, & \text{without blocks}\\
        \frac{\text{Matrix Size}^2 * 4 * \text{\textit{Float} Size}}{\text{Run Time}}, & \text{with blocks}
    \end{cases} \quad
    \text{[GB/s]}
\end{equation}

where $B_r$ and $B_w$ are the number of bytes read and written, respectively, and \textit{Float} size is the size of a %
\texttt{float} variable in bytes. \texttt{Times} is the number of times the matrix is read and written during the execution %
of the algorithm. In the case of the algorithm without blocks, the matrix is read and written three times: once to subdivide %
the matrix, once to transpose it, and once to recompose the final matrix. In the case of the algorithm with blocks, the matrix %
is read and written once more than the algorithm without blocks, caused by the additional step of subdividing into smaller blocks.

The results obtained are summarized in Table \ref{table:strong-scaling} and \ref{table:weak-scaling}. %

\begin{tabularx}{\textwidth}{@{} c c c c c c @{}}
    \caption{\label{table:strong-scaling}Strong scaling}\\
    \toprule
        \textbf{Size} & \textbf{Cores} & \multicolumn{2}{c}{\textbf{Normal}} & \multicolumn{2}{c}{\textbf{Blocks}}\\
        \cmidrule(lr){3-4}\cmidrule(lr){5-6}
        & & \textbf{Run Time [s]} & \textbf{Bandwidth [GB/s]} & \textbf{Run Time [s]} & \textbf{Bandwidth [GB/s]} \\
    \midrule
    \endhead
        \multirow{4}{*}{512} & 1  &  &  &  &  \\
                             & 4  &  &  &  &  \\
                             & 16 &  &  &  &  \\
                             & 64 &  &  &  &  \\
    \midrule
        \multirow{4}{*}{2048} & 1  &  &  &  &  \\
                              & 4  &  &  &  &  \\
                              & 16 &  &  &  &  \\
                              & 64 &  &  &  &  \\
    \midrule
        \multirow{4}{*}{8192} & 1  &  &  &  &  \\
                              & 4  &  &  &  &  \\
                              & 16 &  &  &  &  \\
                              & 64 &  &  &  &  \\
    \midrule
        \multirow{4}{*}{32768} & 1  &  &  &  &  \\
                               & 4  &  &  &  &  \\
                               & 16 &  &  &  &  \\
                               & 64 &  &  &  &  \\
    \bottomrule
\end{tabularx}
\begin{tabularx}{\textwidth}{@{} c c c c c c @{}}
    \caption{\label{table:weak-scaling}Weak scaling}\\
    \toprule
        \textbf{Ratio} & \textbf{CPUs} & \multicolumn{2}{c}{\textbf{Normal}} & \multicolumn{2}{c}{\textbf{Blocks}}\\
        \cmidrule(lr){3-4}\cmidrule(lr){5-6}
        & & \textbf{Run Time [s]} & \textbf{Bandwidth [GB/s]} & \textbf{Run Time [s]} & \textbf{Bandwidth [GB/s]} \\
    \midrule
    \endhead
        \multirow{4}{*}{64} & 1  & 0.00034 & 1.45 & 0.000097 & 0.68 \\
                            & 4  & 0.001141 & 0.69 & 0.001797 & 0.58 \\
                            & 16 & 0.013206 & 0.95 & 0.010417 & 1.61 \\
                            & 64 & 0.218325 & 0.92 & 0.171367 & 1.57 \\
    \midrule
        \multirow{4}{*}{128} & 1  & 0.000111 & 1.77 & 0.000336 & 0.78 \\
                             & 4  & 0.002095 & 1.50 & 0.003130 & 1.34 \\
                             & 16 & 0.039099 & 1.29 & 0.042607 & 1.58 \\
                             & 64 & 0.650468 & 1.24 & 0.684192 & 1.57 \\
    \midrule
        \multirow{4}{*}{256} & 1  & 0.000441 & 1.78 & 0.001309 & 0.80 \\
                             & 4  & 0.006750 & 1.86 & 0.010760 & 1.56 \\
                             & 16 & 0.107758 & 1.87 & 0.158777 & 1.69 \\
                             & 64 & 5.574380 & 0.58 & 2.789386 & 1.54 \\
    \midrule
        \multirow{4}{*}{512} & 1  & 0.002263 & 1.39 & 0.005754 & 0.73 \\
                             & 4  & 0.025396 & 1.98 & 0.043814 & 1.53 \\
                             & 16 & 0.659142 & 1.22 & 0.636953 & 1.69 \\
                             & 64 & - & - & - & 1.51 \\
    \bottomrule
\end{tabularx}

The results show that the parallel algorithm outperforms the serial version for all tested configurations. The speedup increases %
with the number of processors, reaching a maximum of 15.5x for 16 processors. The results also show that the parallel algorithm %
is not perfectly scalable, as the speedup does not increase linearly with the number of processors. This is likely due to the %
communication overhead introduced by the \texttt{MPI\_Scatterv} and \texttt{MPI\_Gatherv} functions, which are used to distribute %
and collect the matrix blocks. The results also show that the rectangular block version of the algorithm performs better than the %
square block version. This is likely due to the fact that the rectangular block version does not require the number of processors %
to be a perfect square, thereby allowing for a more efficient utilization of available resources. The rectangular block version %
also outperforms the square block version in terms of bandwidth, as shown in Table \ref{table:strong-scaling} and Table \ref{table:weak-scaling}.
